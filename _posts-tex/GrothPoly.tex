\documentclass{article}

\ifdefined\HCode
  \def\pgfsysdriver{pgfsys-dvisvgm4ht.def}
\fi 
\usepackage{ebproof}
\usepackage{tikz}
\usepackage{tikz-cd}

\title{Syntax, semantics and fibrations}
\author{Brandon Hewer}

\begin{document}
  \maketitle

  \noindent
  To begin this construction, I present the humble (equilateral) triangle.

  \begin{center}
    \begin{pgfpicture}
      \pgfpathrectangle{\pgfpointorigin}{\pgfqpoint{80.0000bp}{69.0000bp}}
      \pgfusepath{use as bounding box}
      \begin{pgfscope}
        \definecolor{fc}{rgb}{0.0000,0.0000,0.0000}
        \pgfsetfillcolor{fc}
        \pgfpathqmoveto{80.0000bp}{0.0000bp}
        \pgfpathqlineto{40.0000bp}{69.2820bp}
        \pgfpathqlineto{0.0000bp}{0.0000bp}
        \pgfpathqlineto{80.0000bp}{0.0000bp}
        \pgfpathclose
        \pgfusepathqfill
      \end{pgfscope}
    \end{pgfpicture}
  \end{center}

  \noindent
  The first step of constructing Sierpi\'nski's triangle is to divide our triangle into four congruent, equilateral triangles and remove the central triangle.

  \begin{center}
    \begin{pgfpicture}
      \pgfpathrectangle{\pgfpointorigin}{\pgfqpoint{80.0000bp}{69.0000bp}}
      \pgfusepath{use as bounding box}
      \begin{pgfscope}
        \definecolor{fc}{rgb}{0.0000,0.0000,0.0000}
        \pgfsetfillcolor{fc}
        \pgfpathqmoveto{80.0000bp}{0.0000bp}
        \pgfpathqlineto{60.0000bp}{34.6410bp}
        \pgfpathqlineto{40.0000bp}{0.0000bp}
        \pgfpathqlineto{80.0000bp}{0.0000bp}
        \pgfpathclose
        \pgfusepathqfill
      \end{pgfscope}
      \begin{pgfscope}
        \definecolor{fc}{rgb}{0.0000,0.0000,0.0000}
        \pgfsetfillcolor{fc}
        \pgfpathqmoveto{40.0000bp}{0.0000bp}
        \pgfpathqlineto{20.0000bp}{34.6410bp}
        \pgfpathqlineto{0.0000bp}{0.0000bp}
        \pgfpathqlineto{40.0000bp}{0.0000bp}
        \pgfpathclose
        \pgfusepathqfill
      \end{pgfscope}
      \begin{pgfscope}
        \definecolor{fc}{rgb}{0.0000,0.0000,0.0000}
        \pgfsetfillcolor{fc}
        \pgfpathqmoveto{60.0000bp}{34.6410bp}
        \pgfpathqlineto{40.0000bp}{69.2820bp}
        \pgfpathqlineto{20.0000bp}{34.6410bp}
        \pgfpathqlineto{60.0000bp}{34.6410bp}
        \pgfpathclose
        \pgfusepathqfill
      \end{pgfscope}
    \end{pgfpicture}
  \end{center}

\end{document}
